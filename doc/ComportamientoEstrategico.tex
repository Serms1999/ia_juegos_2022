 \section{Comportamiento Estratégico}
 
 \subsection{Modos de comportamiento}
 
 El comportamiento estratégico implementado a nivel de bando/grupo se ha hecho de forma que tenemos varios modos.
 
 \begin{itemize}
     \item Ataque: El bando que esté en modo ataque irá a por la base enemiga y todo aquel que se encuentre por su paso lo tratará de eliminar.
     \item Defensa: Este modo es para que el equipo seleccionado se centre en defender su base de los enemigos y priorizarán también la cura de los personajes.
     \item Guerra Total: En este modo ambos equipos pasan al ataque y ganará el primero en conquistar la base enemiga
 \end{itemize}
 
La manera en la que se adapta el comportamiento según el modo es sencilla basándonos en el uso de la librería Behaviour Trees \cite{uniBT}. Tenemos la clase \texttt{State} que recoge todos los modos de juego y cada personaje tendrá un atributo de este tipo.

\lstinputlisting[linerange=5-10, firstnumber=5]{\ScriptsPath/Tactica/State.cs}

El cambio de modo es a través de los botones, donde cada botón llamará al método de la clase \texttt{GameController} correspondiente:

\lstinputlisting[linerange=105-108, firstnumber=105]{\ScriptsPath/GameController.cs}

\lstinputlisting[linerange=113-116, firstnumber=113]{\ScriptsPath/GameController.cs}

\lstinputlisting[linerange=121-124, firstnumber=121]{\ScriptsPath/GameController.cs}

\lstinputlisting[linerange=129-132, firstnumber=129]{\ScriptsPath/GameController.cs}

\lstinputlisting[linerange=137-141, firstnumber=137]{\ScriptsPath/GameController.cs}

El método que contiene toda la lógica de los cambios de modo es \texttt{SwitchTeamMode}, que se encarga de buscar dentro de todos los NPCs que tenemos  aquellos que coinciden con el equipo pasado por parámetro y el estado se modificará al que se tiene como segundo parámetro.

\lstinputlisting[linerange=89-100, firstnumber=89]{\ScriptsPath/GameController.cs}
