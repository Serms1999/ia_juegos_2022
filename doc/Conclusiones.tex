\section{Conclusiones}

Para concluir el documento hemos visto conveniente hacer una exposición de conclusiones a nivel individual y de grupo sobre la práctica ya que creemos que pueden ser tenidas en cuenta en el futuro.

Lo primero a comentar sería que no se ha llevado un registro estricto de las horas dedicadas a la práctica en general, pero si es cierto que se nota la envergadura de esta y la alta dedicación que requiere, ya que en esta convocatoria que hemos tenido un volumen menor de asignaturas el resultado de la práctica a mejorado respecto a lo visto en convocatorias anteriores donde nos tocaba compartir el tiempo con otras asignaturas, viéndose afectada la calidad del trabajo final.

La primera parte en un comienzo nos resultó complicada ya que requería entender bien los comportamientos que se pedían y sobretodo una vez se pensaba que los \textit{Steerings} estaban implementados surgían con bastante frecuencia \textit{bugs} y errores inesperado, llegando a la conclusión de que en la primera parte se puede llegar a dedicar una cantidad de tiempo a desarrollar todos los comportamiento y aproximadamente la mitad a corregir fallos y pulir el comportamiento. Es cierto que estas correcciones son debidas a la adaptación del código, pero es de bastante ayuda en esta primera parte la cantidad de documentación que se puede encontrar y sobre la que nos podemos apoyar.

En cuanto a la segunda parte se ha dedicado bastante tiempo a la hora de crear los elementos gráficos, ya sea buscar assets, personajes, implementar la UI y demás elementos gráficos. La mayor dificultad encontrada en esta parte ha sido sin duda el arranque ya que hemos tenido que dedicar bastante tiempo a discutir lo que se tenía que hacer porque andábamos un tanto perdidos en cuanto a lo que teníamos que realizar y ha sido una tónica constante en el desarrollo de esta parte. Es cierto que una vez dimos con la tecla y conseguimos plasmar todas las ideas que teníamos, la sensación final que se nos queda con esta segunda parte es bastante buena ya que hemos podido ver cómo se comportan los agentes acorde a nuestro árboles de comportamiento.

La conclusión final que sacamos es que es muy complicado ajustar todos estos parámetros para que el comportamiento final sea el que se desea, ya que en algo con tantos niveles de complejidad, donde hay muchos factores que influyen en el comportamiento siempre se puede encontrar algo que no se comporta al 100\% como se esperaba.