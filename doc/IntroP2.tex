\part{Bloque 2}

 Una vez implementado el comportamiento basado  en el movimiento de unidades se pasará a implementar distintos comportamientos propios de una IA táctica dentro de un entorno de juego de guerra en tiempo real.  En este bloque se hará un repaso por los distintos pasos y decisiones tomadas para superar los distintos apartados (obligatorios y opcionales) necesarios para implementar estos elementos.

Para esta práctica se utilizará todo lo implementado en la primera parte de la asignatura relacionado con movimiento de personajes, formaciones y colisiones.

El sistema de combate implementado consta de distintos tipos de unidades y terrenos, donde cada unidad tendrá distintos valores tanto de vida, ataque, alcance (cuerpo a cuerpo y a distancia) y velocidad de movimiento que dependerá en muchos caso del terreno por el que se mueve el personaje. El mapa se ha implementado de forma manual como un grid, teniendo dos tipos de mapas. Por un lado el mapa convencional que consta de distintos tipos de terrenos, puntos de interés, bases, personajes, etc.; y por otro lado un mapa que mostrará las distintas influencias en forma de `mapa de calor'.

Además  existirán distintas zonas de interés táctico en el mapa como pueden ser puntos de curación, zonas de paso entre `regiones', bases o puntos intermedios de cruces de caminos. Cada una de estas zonas de interés tendrá un tipo de comportamiento dependiendo del equipo, por ejemplo, las bases solo podrán ser tomadas por personajes enemigos.

El movimiento de los personajes del juego estará condicionado por distintos factores que se irán comentando en las siguientes secciones.

Tendremos en cuenta el tipo de terreno, el tipo de personaje, la interacción entre los distintos personajes con los tipos de terrenos que marcará un comportamiento distinto con atributos cambiantes como puede ser la velocidad de los personajes y por último la influencia del mapa que se utilizará también para el comportamiento de los personajes.

