\section{Sistema de combate}

El sistema de combate consiste en simples ataques entre los distintos personajes donde la evolución del combate dependerá de las distintas características de las unidades. Estos para atacar necesitan estar en rango de ataque, teniendo cada uno un rango específico, por ejemplo, los arqueros tienen más rango que el resto. Veamos los atributos que toman parte en el combate y los valores que toman estos para cada personaje:

\begin{itemize}
    \item \texttt{\_baseDamage:} Daño base de la unidad, varía según el tipo.
    \item \texttt{\_attackRange:} Distancia máxima a la que se considera que un objetivo está cerca, y por tanto se le puede atacar.
    \item \texttt{\_attackSpeed:} Velocidad a la que ataque el agente.
    \item \texttt{\_hpMax:} Puntos de vida máximos del agente.
    \item \texttt{\_hpCurrent:} Puntos de vida actuales del agente.
    \item \texttt{healSpeed:} Velocidad a la que recupera vida el agente.
    \item \texttt{captureSpeed:} Velocidad de captura de la base enemiga.
\end{itemize}

En la siguiente tabla podemos ver cómo varían estos atributos según la unidad:
\begin{table}[H]
    \centering
    \begin{tabular}{|c|c|c|c|c|}
       \hline        
       \textbf{Unidad} & Daño base & Rango de ataque & Velocidad de ataque & Vida máxima \\
        \hline
        Lancero & 40 & 6 & 4 & 250 \\
        \hline
        Infantería & 10 & 6 & 4 & 200 \\
        \hline
        Caballería & 30 & 6 & 4 & 130 \\
        \hline
        Arquero & 20 & 14 & 4 & 100 \\
        \hline
    \end{tabular}
    \caption{Tabla de Influencias}
\end{table}

Una vez estamos en rango de ataque el agente hará un daño aleatorio entre su rango, este podrá ir entre el \qtyrange[range-phrase=~~--~, range-units=single]{80}{100}{\percent} del mismo. Tras esto, para seguir atacando el NPC tendrá que esperar según su velocidad de ataque.

Es el método \texttt{AttackEnemy} el que incluye la lógica de combate entre unidades, donde podemos ver cómo se obtiene un valor aleatorio entre 0.8 y 1 que se usará como factor multiplicativo para el ataque base del agente. Posteriormente se llama al método \texttt{GetDamage} del enemigo, para que reduzca sus puntos de vida (en el caso de estar en su punto de curación el agente no recibirá daño). 

\lstinputlisting[linerange=432-456, firstnumber=432]{\ScriptsPath/Steering/NPC/AgentNPC.cs}

Una vez se ha terminado un combate hay 2 posibilidades. Si sigue vivo el personaje, seguirá con su funcionamiento definido; podrá ir a curarse, seguir atacando, defender, etc. La otra posibilidad que queda es que ese personaje haya muerto, si este es el caso reaparecerá después de un tiempo en su base.

\subsection{Condición de victoria}

En cuanto a las condiciones de victoria se ha decidido implementar solamente una, que será la de conseguir capturar la base enemiga ya que  se supone que es la más completa a la hora de mostrar el funcionamiento de todo lo implementado en el juego pudiendo probar los comportamientos de pathFinding, influencia y el ataque entre unidades.

Será el método \texttt{CaptureEnemyBase} de la clase \texttt{AgentNPC} el que implemente este comportamiento, donde podemos ver cómo se realizan las mismas comprobaciones para que se apliquen los puntos de captura cada X segundos y  dónde se llama al método de la base enemiga que le añade puntos de captura.

\lstinputlisting[linerange=706-727, firstnumber=706]{\ScriptsPath/Steering/NPC/AgentNPC.cs}

Una vez completados los puntos de captura saltará la condición de victoria, se pausará el juego y aparecerá una ventana indicando cuál ha sido el equipo ganador. 