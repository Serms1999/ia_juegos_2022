\subsubsection{Path Following}

Este steering hace que el agente siga un camino marcado por puntos (nodos).

\begin{figure}[H]
    \centering
    \resizebox{0.25\textwidth}{!}{
        \begin{tikzpicture}
    \node[circle, minimum size=1cm, draw, dotted] (1) {};
    \node[circle, minimum size=1cm, draw, dotted] (2) at (1,1) {};
    \node[circle, minimum size=1cm, draw, dotted] (3) at (1, -2) {};
    \node[circle, minimum size=1cm, draw, dotted] (4) at (-2, -2) {};
    \node[circle, minimum size=1cm, draw, dotted] (5) at (-3, 2){};
    
    \draw (1.center) -- (2.center);
    \draw (2.center) -- (3.center);
    \draw (3.center) -- (4.center);
    \draw (4.center) -- (5.center);
    \draw[dashed] (5.center) -- (1.center);
    
    \node[scale=0.3] (1) {\includegraphics{images/sv_icon_dot14_pix16_gizmo}};
    \node[scale=0.3] (2) at (1,1) {\includegraphics{images/sv_icon_dot14_pix16_gizmo}};
    \node[scale=0.3] (3) at (1, -2) {\includegraphics{images/sv_icon_dot14_pix16_gizmo}};
    \node[scale=0.3] (4) at (-2, -2) {\includegraphics{images/sv_icon_dot14_pix16_gizmo}};
    \node[scale=0.3] (5) at (-3, 2){\includegraphics{images/sv_icon_dot14_pix16_gizmo}};
\end{tikzpicture}
    
    }
    \caption{Esquema de un camino marcado}
    \label{fig:pathFollowing}
\end{figure}

Hay varias formas de recorrer el mismo camino:
\begin{enumerate}
    \item Llegar desde el inicio hasta el final y permanecer ahí.
    \item Llegar desde el inicio hasta el final y darse la vuelta, volviendo a hacer el camino de forma inversa, y continuar así.
    \item Hacer el camino de forma cíclica, es decir, una vez se llega al final el personaje vuelve al nodo de inicio directamente y vuelve a repetir el camino. Esta forma de hacer el recorrido sólo es posible en camino cerrados.
\end{enumerate}

En la Fig. \ref{fig:pathFollowing} se ilustra como se define un camino, como una serie de nodos unidos. Opcionalmente hay un camino que une principio con final, esto se ilustra con un línea discontinua.

Para implementar este steering se han creado dos clases, \texttt{Path} y \texttt{Waypoint}. La clase \texttt{Waypoint} es una estructura que tiene información sobre la posición y el radio de un nodo del camino. El radio nos indica si estamos lo suficientemente cerca del nodo como para considerar si ya lo hemos visitado.

Por otro lado, la clase \texttt{Path} tiene guardados todos los nodos (waypoints) del camino. También tiene información sobre el orden para recorrerlo o si el camino es circular.

El steering en sí tiene una implementación sencilla. Lo primero es comprobar si ya hemos llegado al nodo hacia el que nos dirigíamos, en caso afirmativo pasamos al siguiente.

\lstinputlisting[linerange=33-40, firstnumber=33]{\ScriptsPath/Steering/Delegados/PathFollowing.cs}

Lo siguiente es que hacer dependiendo de la forma de recorrer el camino, solo se aplica en los nodos extremos del camino.

\lstinputlisting[linerange=42-74, firstnumber=42]{\ScriptsPath/Steering/Delegados/PathFollowing.cs}

Por último, una vez tenemos el nodo al que ir, creamos un agente auxiliar y delegamos en el Seek.

\lstinputlisting[linerange=79-87, firstnumber=79]{\ScriptsPath/Steering/Delegados/PathFollowing.cs}


