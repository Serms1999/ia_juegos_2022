\subsubsection{Wander}

Este steering hace que el agente que lo implementa se mueva erráticamente por el escenario.

Para lograr este comportamiento haremos uso de una estructura como la que se puede ver en la Fig. \ref{fig:wander}. Esta estructura está compuesta por una circunferencia sitiada a una distancia del agente.

\begin{figure}[H]
    \centering
    \begin{tikzpicture}
    \node[isosceles triangle, draw, scale=0.7, color= BurntOrange, fill= BurntOrange!70] (1) {};
    \node[circle, draw, minimum size=2cm, dashed] (2) [right = 2cm of 1] {};
    \draw[fill=green!20] (3.35,0) --  +(60:1) arc(60:-60:1) -- cycle;
    \node (3) at (4, 0.75) {$\bigoplus$};
    \node (4) at (1.5, -0.2) {\tiny\texttt{wanderOffset}};
    \node (5) at (3.7, -0.3) {\tiny\texttt{wanderRadius}};
    \node (6) at (5.5, 0.2) {\tiny\texttt{wanderRate}};
    
    \draw[dashed] (1) -- (2.center);
    \draw[-stealth] (0.67, 0.12) -- (2.4, 0.42);
    \draw[stealth-stealth] (2.435, -0.4) -- (2.center);
    \draw[-stealth] (6.west) -- (4.4, 0.2);
\end{tikzpicture}
    \caption{Esquema del steering Wander}
    \label{fig:wander}
\end{figure}

Como vemos existen 3 parámetros que podemos modificar para cambiar el comportamiento de este steering.
\begin{itemize}
    \item \texttt{wanderOffset}: la distancia entre el centro de la circunferencia y el agente.
    \item \texttt{wanderRadius}: el radio de la circunferencia.
    \item \texttt{wanderRate}: el rango de ángulos que podemos generar automáticamente.
\end{itemize}
 
La implementación de este steering consiste en generar un punto en la circunferencia, que se encuentre en el rango de ángulos $(-\texttt{wanderRate}, \texttt{wanderRate})$. Con esto conseguimos que el agente se vaya desplazando hacia delate hacia puntos aleatorios.

La distancia entre el agente y la circunferencia modificará como de brusco es el giro y, por tanto, el agente avanzará menos y cambiará más de trayectoria.

Lo primero que hacemos es calcular la posición de la circunferencia en la que estará el punto.

\lstinputlisting[linerange=46-52, firstnumber=46]{\ScriptsPath/Steering/Delegados/Wander.cs}

A continuación, calculamos el centro de la circunferencia y delegamos en el Face.

\lstinputlisting[linerange=54-61, firstnumber=54]{\ScriptsPath/Steering/Delegados/Wander.cs}
